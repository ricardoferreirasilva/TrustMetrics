%%%%%%%%%%%%%%%%%%%%%%%%%%%%%%%%%%%%%%%%%
% Short Sectioned Assignment
% LaTeX Template
% Version 1.0 (5/5/12)
%
% This template has been downloaded from:
% http://www.LaTeXTemplates.com
%
% Original author:
% Frits Wenneker (http://www.howtotex.com)
%
% License:
% CC BY-NC-SA 3.0 (http://creativecommons.org/licenses/by-nc-sa/3.0/)
%
%%%%%%%%%%%%%%%%%%%%%%%%%%%%%%%%%%%%%%%%%

%----------------------------------------------------------------------------------------
%	PACKAGES AND OTHER DOCUMENT CONFIGURATIONS
%----------------------------------------------------------------------------------------

\documentclass[paper=a4, fontsize=11pt]{scrartcl} % A4 paper and 11pt font size

\usepackage[T1]{fontenc} % Use 8-bit encoding that has 256 glyphs
\usepackage{fourier} % Use the Adobe Utopia font for the document - comment this line to return to the LaTeX default
\usepackage[english]{babel} % English language/hyphenation
\usepackage{amsmath,amsfonts,amsthm} % Math packages
\usepackage{graphicx}
\usepackage{tikz}
\usetikzlibrary{arrows}
\usetikzlibrary{positioning}

\usepackage{lipsum} % Used for inserting dummy 'Lorem ipsum' text into the template

\usepackage{sectsty} % Allows customizing section commands
\allsectionsfont{\centering \normalfont\scshape} % Make all sections centered, the default font and small caps

\usepackage{fancyhdr} % Custom headers and footers
\pagestyle{fancyplain} % Makes all pages in the document conform to the custom headers and footers
\fancyhead{} % No page header - if you want one, create it in the same way as the footers below
\fancyfoot[L]{} % Empty left footer
\fancyfoot[C]{} % Empty center footer
\fancyfoot[R]{\thepage} % Page numbering for right footer
\renewcommand{\headrulewidth}{0pt} % Remove header underlines
\renewcommand{\footrulewidth}{0pt} % Remove footer underlines
\setlength{\headheight}{13.6pt} % Customize the height of the header

\numberwithin{equation}{section} % Number equations within sections (i.e. 1.1, 1.2, 2.1, 2.2 instead of 1, 2, 3, 4)
\numberwithin{figure}{section} % Number figures within sections (i.e. 1.1, 1.2, 2.1, 2.2 instead of 1, 2, 3, 4)
\numberwithin{table}{section} % Number tables within sections (i.e. 1.1, 1.2, 2.1, 2.2 instead of 1, 2, 3, 4)

\setlength\parindent{0pt} % Removes all indentation from paragraphs - comment this line for an assignment with lots of text

%----------------------------------------------------------------------------------------
%	TITLE SECTION
%----------------------------------------------------------------------------------------

\newcommand{\horrule}[1]{\rule{\linewidth}{#1}} % Create horizontal rule command with 1 argument of height

\title
{	
\normalfont \normalsize 
\includegraphics[scale=0.7]{feup.png}
\horrule{0.5pt} \\[0.4cm] % Thin top horizontal rule
\huge Project Management with Computational Trust Models in a Multi-Agent Environment\\
\horrule{2pt} \\[0.5cm] % Thick bottom horizontal rule
}

\author{
	Ricardo Ferreira da Silva\\
	\texttt{up201305163@fc.up.pt}
	\and
	Gustavo de Castro Nogueira Pinto\\
	\texttt{up201302828@fe.up.pt}
	\and
	Tiago Filipe Abreu Figueiredo\\
	\texttt{ei12069@fe.up.pt}
}
\date{\normalsize\today} % Today's date or a custom date
\begin{document}

\maketitle % Print the title

%----------------------------------------------------------------------------------------
%	PROBLEM 1
%----------------------------------------------------------------------------------------
\newpage

\tableofcontents

\newpage
\section{Objectives}
% and objectives
%------------------------------------------------
A project, in the context of operations research consists of a set of interdependent tasks, a start state and an end state. Each of these tasks has a certain duration and it is required that the tasks preceding it are completed before we can execute it. The start state gives us access to the initial tasks (tasks without precedences), and the final state is reached when all tasks are completed, which means the project itself was completed.

% Maybe example
Expanding upon this context, lets say we are in an simple entrepreneurial setting. A company will certainly have a project which it intends to complete and it possesses a set of workers, each one with a different set of skills and different degrees of proficiency at those skills. There will also exist a very special type of worker, that is, the manager. The manager has the crucial function of evaluating all its subordinate workers and assigning them to the tasks they will be most useful at, with the ultimate objective of finishing the overall project in the shortest amount of time.

Our objective, is to model this problem in an multi-agent environment setting and use computational trust model in order to obtain an evaluation of these workers via a manager, thus obtaining an approximation of the information needed to make the correct choices in terms of allocation.

\section{Specification}
Just as it was described in the previous chapter, the objective of this application is to simulate the development of a project. This project consists of a set of interdependent tasks that must be concluded, a a manager responsible of making decisions and a set of workers, whose ratings are not fully known by the manager.

\subsection{Project Structure}

\subsection{Agents}
In order to model this problem in a multi-agent environment we decided to design three of the main entities in the problem as agents: Manager, Worker and Task. The worker and the task are mostly reactive agents since all their actions are triggered via a manager order, with an exception we will later discuss. The manager is a BDI agent and he is the true "mastermind" so to say behind the project. He has the final objective of finishing the project in the shortest amount of time and he must get to know the worker agents properly so he can effectively allocate them to the available tasks.

\subsubsection{Worker}

\subsubsection{Task}

\subsubsection{Manager}
The manager is a "Beliefs-Desires-Intentions" type agent that is responsible for allocating the available workers to available tasks with the ultimate goal of finishing the project in the shortest amount of time. The manager has knowledge of all tasks and of all workers. Regarding workers, the manager can order them to work on a task and he does this by deciding if a worker is useful at working in a particular task by comparing his skills and ratings with the skills required by the task. However,
\subsection{Interaction Protocols}

\section{Development}

\subsection{Platform}

\subsection{Modules}

\subsection{Implementation}

\section{Experiences}

\section{Conclusions}

\section{Improvement}

\end{document}
